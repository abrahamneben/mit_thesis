% -*-latex-*-
% 
% For questions, comments, concerns or complaints:
% thesis@mit.edu
% 
%
% $Log: cover.tex,v $
% Revision 1.8  2008/05/13 15:02:15  jdreed
% Degree month is June, not May.  Added note about prevdegrees.
% Arthur Smith's title updated
%
% Revision 1.7  2001/02/08 18:53:16  boojum
% changed some \newpages to \cleardoublepages
%
% Revision 1.6  1999/10/21 14:49:31  boojum
% changed comment referring to documentstyle
%
% Revision 1.5  1999/10/21 14:39:04  boojum
% *** empty log message ***
%
% Revision 1.4  1997/04/18  17:54:10  othomas
% added page numbers on abstract and cover, and made 1 abstract
% page the default rather than 2.  (anne hunter tells me this
% is the new institute standard.)
%
% Revision 1.4  1997/04/18  17:54:10  othomas
% added page numbers on abstract and cover, and made 1 abstract
% page the default rather than 2.  (anne hunter tells me this
% is the new institute standard.)
%
% Revision 1.3  93/05/17  17:06:29  starflt
% Added acknowledgements section (suggested by tompalka)
% 
% Revision 1.2  92/04/22  13:13:13  epeisach
% Fixes for 1991 course 6 requirements
% Phrase "and to grant others the right to do so" has been added to 
% permission clause
% Second copy of abstract is not counted as separate pages so numbering works
% out
% 
% Revision 1.1  92/04/22  13:08:20  epeisach

% NOTE:
% These templates make an effort to conform to the MIT Thesis specifications,
% however the specifications can change.  We recommend that you verify the
% layout of your title page with your thesis advisor and/or the MIT 
% Libraries before printing your final copy.
\title{Tools, techniques, and early results in studies of 21\,cm, Lyman-$\alpha$, and H-$\alpha$ emission from the cosmic dawn}

\author{Abraham Richard Neben}
% If you wish to list your previous degrees on the cover page, use the 
% previous degrees command:
       \prevdegrees{A.B., University of Chicago (2011)}
% You can use the \\ command to list multiple previous degrees
%       \prevdegrees{B.S., University of California (1978) \\
%                    S.M., Massachusetts Institute of Technology (1981)}
\department{Department of Physics}

% If the thesis is for two degrees simultaneously, list them both
% separated by \and like this:
\degree{Doctor of Philosophy in Physics}
%\degree{Bachelor of Science in Computer Science and Engineering}

% As of the 2007-08 academic year, valid degree months are September, 
% February, or June.  The default is June.
\degreemonth{May}
\degreeyear{2017}
\thesisdate{May XX, 2017}

%% By default, the thesis will be copyrighted to MIT.  If you need to copyright
%% the thesis to yourself, just specify the `vi' documentclass option.  If for
%% some reason you want to exactly specify the copyright notice text, you can
%% use the \copyrightnoticetext command.  
%\copyrightnoticetext{\copyright IBM, 1990.  Do not open till Xmas.}

% If there is more than one supervisor, use the \supervisor command
% once for each.
\supervisor{Jacqueline N. Hewitt}{Professor of Physics}

% This is the department committee chairman, not the thesis committee
% chairman.  You should replace this with your Department's Committee
% Chairman.
\chairman{Nergis Mavalvala}{Professor of Physics\\Associate Department Head for Education}

% Make the titlepage based on the above information.  If you need
% something special and can't use the standard form, you can specify
% the exact text of the titlepage yourself.  Put it in a titlepage
% environment and leave blank lines where you want vertical space.
% The spaces will be adjusted to fill the entire page.  The dotted
% lines for the signatures are made with the \signature command.
\maketitle

% The abstractpage environment sets up everything on the page except
% the text itself.  The title and other header material are put at the
% top of the page, and the supervisors are listed at the bottom.  A
% new page is begun both before and after.  Of course, an abstract may
% be more than one page itself.  If you need more control over the
% format of the page, you can use the abstract environment, which puts
% the word "Abstract" at the beginning and single spaces its text.

%% You can either \input (*not* \include) your abstract file, or you can put
%% the text of the abstract directly between the \begin{abstractpage} and
%% \end{abstractpage} commands.

% First copy: start a new page, and save the page number.
\cleardoublepage
% Uncomment the next line if you do NOT want a page number on your
% abstract and acknowledgments pages.
% \pagestyle{empty}
\setcounter{savepage}{\thepage}
\begin{abstractpage}
% $Log: abstract.tex,v $
% Revision 1.1  93/05/14  14:56:25  starflt
% Initial revision
% 
% Revision 1.1  90/05/04  10:41:01  lwvanels
% Initial revision
% 
%
%% The text of your abstract and nothing else (other than comments) goes here.
%% It will be single-spaced and the rest of the text that is supposed to go on
%% the abstract page will be generated by the abstractpage environment.  This
%% file should be \input (not \include 'd) from cover.tex.

This thesis presents new instrumentation and analysis techniques for observations of redshifted 21\,cm and Lyman-$\alpha$, and H-$\alpha$  emission from the Epoch of Reionization (EOR), as well as early science results from their application to real data. We develop a nearly all-sky, high dynamic range antenna measurement system and use it to characterize prototype antenna elements for the Murchison Widefield Array (MWA) and the Hydrogen Epoch of Reionization Array (HERA). We measure the model deviations at large zenith angles where the beam response crucially sets the amount of foreground power with high apparent frequency dependence in interferometer observations. We then characterize the antenna-to-antenna variation over the MWA with laboratory measurements, and show that neglecting this effect results in severe foreground mis-subtraction.

We build on the optimal quadratic power spectrum analysis of Liu \& Tegmark and Dillon, Liu, \& Tegmark by estimating the covariance from the data itself in a suitably constrained way. We apply this method to 3\,hours of MWA data and set a limit on the 21\,cm EOR power spectrum of $\Delta(k) < 192$\,mK (95\%) at comoving scale $k = 0.18$\,$h$\,Mpc$^{-1}$ and $z = 6.8$, consistent with limits from other experiments. Then, to extend the scientific return of future 21\,cm detections, we perform a foreground and sensitivity analysis for 2D (i.e., broad band) 21\,cm--Ly$\alpha$ and 21\,cm--H$\alpha$ cross correlation experiments. Using data from the MWA, the Wide-field Infrared Survey Explorer (WISE), and the Asteroid Terrestrial-impact Last Alert System (ATLAS) we find percent-level flux correlations between radio and near-infrared foregrounds, consistent with geometric effects. We use the optimal quadratic estimator to set a first limit of $\Delta^2<181\,(\text{kJy/sr}\cdot \text{mK})$ (95\%) at $\ell=800$ on the broad band angular 21\,cm--Ly-$\alpha$ cross spectrum at $z\sim7$. We show that in the near term, higher resolution radio and near-infrared surveys such as LOFAR, Hubble, Dark Energy Survey can start to probe optimistic models of the 2D 21\,cm--Ly$\alpha$ EOR cross spectrum. 


\end{abstractpage}

% Additional copy: start a new page, and reset the page number.  This way,
% the second copy of the abstract is not counted as separate pages.
% Uncomment the next 6 lines if you need two copies of the abstract
% page.
% \setcounter{page}{\thesavepage}
% \begin{abstractpage}
% % $Log: abstract.tex,v $
% Revision 1.1  93/05/14  14:56:25  starflt
% Initial revision
% 
% Revision 1.1  90/05/04  10:41:01  lwvanels
% Initial revision
% 
%
%% The text of your abstract and nothing else (other than comments) goes here.
%% It will be single-spaced and the rest of the text that is supposed to go on
%% the abstract page will be generated by the abstractpage environment.  This
%% file should be \input (not \include 'd) from cover.tex.

This thesis presents new instrumentation and analysis techniques for observations of redshifted 21\,cm and Lyman-$\alpha$, and H-$\alpha$  emission from the Epoch of Reionization (EOR), as well as early science results from their application to real data. We develop a nearly all-sky, high dynamic range antenna measurement system and use it to characterize prototype antenna elements for the Murchison Widefield Array (MWA) and the Hydrogen Epoch of Reionization Array (HERA). We measure the model deviations at large zenith angles where the beam response crucially sets the amount of foreground power with high apparent frequency dependence in interferometer observations. We then characterize the antenna-to-antenna variation over the MWA with laboratory measurements, and show that neglecting this effect results in severe foreground mis-subtraction.

We build on the optimal quadratic power spectrum analysis of Liu \& Tegmark and Dillon, Liu, \& Tegmark by estimating the covariance from the data itself in a suitably constrained way. We apply this method to 3\,hours of MWA data and set a limit on the 21\,cm EOR power spectrum of $\Delta(k) < 192$\,mK (95\%) at comoving scale $k = 0.18$\,$h$\,Mpc$^{-1}$ and $z = 6.8$, consistent with limits from other experiments. Then, to extend the scientific return of future 21\,cm detections, we perform a foreground and sensitivity analysis for 2D (i.e., broad band) 21\,cm--Ly$\alpha$ and 21\,cm--H$\alpha$ cross correlation experiments. Using data from the MWA, the Wide-field Infrared Survey Explorer (WISE), and the Asteroid Terrestrial-impact Last Alert System (ATLAS) we find percent-level flux correlations between radio and near-infrared foregrounds, consistent with geometric effects. We use the optimal quadratic estimator to set a first limit of $\Delta^2<181\,(\text{kJy/sr}\cdot \text{mK})$ (95\%) at $\ell=800$ on the broad band angular 21\,cm--Ly-$\alpha$ cross spectrum at $z\sim7$. We show that in the near term, higher resolution radio and near-infrared surveys such as LOFAR, Hubble, Dark Energy Survey can start to probe optimistic models of the 2D 21\,cm--Ly$\alpha$ EOR cross spectrum. 


% \end{abstractpage}

\cleardoublepage

\section*{Acknowledgments}

I thank my parents who, for as long as I can remember, encouraged me to take chances, make mistakes, and...well...get messy. When I was five years old I undertook a long running excavation of our driveway's cracking asphalt surface to find out what was underneath. My friend and I chipped it away from the ground, piece by piece, sometimes with shovels but mostly with our bare hands. Ever the supporter of science, instead of reprimanding me for destroying the driveway, my mom simply asked my friend's mom to send a change of clothes next time. When seven-year-old me studied water flow by damming the stream behind our house with rocks, logs, and brush, my parents kindly asked only that I wipe my shoes on the mat before coming inside. 

They encouraged me to ask questions and try to figure things out, even if that meant challenging the conventional wisdom, or challenging them, for that matter. They shared in my enthusiasm when grad school was going well even if they couldn't understand all the words, and they encouraged me when nothing was working. Probably more often the latter than the former. They even stuck by me even when I considered leaving grad school to pursue a fledging career in stand-up comedy.

I thank my advisor, Jackie, who eagerly accepted me into her group, found interesting topics for me to work on, accompanied me on trips to Green Bank, and introduced me to collaborators like Miguel Morales, John Tonry, and Aaron Parsons. It was such a pleasure to learn the surprisingly deep and interesting field of radio astronomy, and to discover that its tools and techniques are so widely applicable outside of astronomy research.

I thank the grad students I worked and procrastinated with who made my time enjoyable and reminded me of why I got into physics in the first place: Jeff Zheng, Aaron Ewall-Wice, Josh Dillon, Adrian Liu, Lu Feng, Adam Beardsley, and Nichole Barry. And Adam Anderson, whom I accidentally followed from Stephenson Elementary School, to Jackson Middle School, to Wilson High School, to the University of Chicago, and finally to MIT. 

My first grad school experience was arguing with Jeff on the empty lower level of the MIT orientation harbor cruise about which theory was `better': quantum mechanics or general relativity. He argued that as quantum mechanics is such a mess of a theory compared to general relativity, it's hard to believe it represents anything fundamental. I argued that the only reason it's so complicated is that it's been honed by a century of experiments, whereas general relativity has only just begun to be tested in detail. That's when I know I would fit right in.


%%%%%%%%%%%%%%%%%%%%%%%%%%%%%%%%%%%%%%%%%%%%%%%%%%%%%%%%%%%%%%%%%%%%%%
% -*-latex-*-
