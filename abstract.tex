% $Log: abstract.tex,v $
% Revision 1.1  93/05/14  14:56:25  starflt
% Initial revision
% 
% Revision 1.1  90/05/04  10:41:01  lwvanels
% Initial revision
% 
%
%% The text of your abstract and nothing else (other than comments) goes here.
%% It will be single-spaced and the rest of the text that is supposed to go on
%% the abstract page will be generated by the abstractpage environment.  This
%% file should be \input (not \include 'd) from cover.tex.
This thesis presents new instrumentation and statistical techniques towards observations of redshifted 21\,cm and ultraviolet emission from the Epoch of Reionization (EOR), as well as early science results from their application to real data. We develop a wide field, high dynamic range antenna measurement system and use it to characterize prototype antenna elements of the Murchison Widefield Array (MWA) and the Hydrogen Epoch of Reionization Array (HERA). We make the first measurements of model deviations at low altitudes, where the beam response sets the amount of foreground power with high fringe rates, most at risk of contaminating cosmological measurements, and quantify the implications for foreground avoidance and subtraction approaches. We extend our analysis of the MWA tile with precision laboratory measurements and make the first predictions of antenna-to-antenna beam variation over the array. 

We then build on the optimal quadratic power spectrum technique of Liu \& Tegmark and Dillon, Liu, \& Tegmark by estimating the foreground covariance from the data itself. By optimizing the foreground downweighting beyond analytic models we achieve competitive limits on EOR emission from three hours of early MWA data. Lastly, we correlate these reduced MWA images with wide field near infrared observations to characterize the foregrounds for a 21\,cm--Infrared correlation experiment, and incorporate these into a sensitivity and experimental design study. Observations of this predicted large scale anticorrelation can provide a cross-check on a purported 21\,cm detection and extend its science returns probing the sources driving ionized bubble growth.

