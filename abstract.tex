% $Log: abstract.tex,v $
% Revision 1.1  93/05/14  14:56:25  starflt
% Initial revision
% 
% Revision 1.1  90/05/04  10:41:01  lwvanels
% Initial revision
% 
%
%% The text of your abstract and nothing else (other than comments) goes here.
%% It will be single-spaced and the rest of the text that is supposed to go on
%% the abstract page will be generated by the abstractpage environment.  This
%% file should be \input (not \include 'd) from cover.tex.

This thesis presents new instrumentation and statistical techniques for observations of redshifted 21\,cm and Lyman-$\alpha$, and H-$\alpha$  emission from the Epoch of Reionization (EOR), as well as early science results from their application to real data. We develop a wide field, high dynamic range antenna measurement system and use it to characterize prototype antenna elements for the Murchison Widefield Array (MWA) and the Hydrogen Epoch of Reionization Array (HERA). We achieve measurements over 30\,dB of beam dynamic range over 65\% of the visible sky, and make the first measurements of model deviations at large zenith angles, where the beam response crucially sets the amount of foreground power with high fringe rate. These foregrounds are most at risk of contaminating cosmological measurements, and we quantify the implications of measured model deviations for foreground avoidance and subtraction approaches. We extend our analysis of the MWA tile with precision laboratory measurements, and study the level of antenna-to-antenna beam variation over the array. 

We then build on the optimal quadratic power spectrum analysis of Liu \& Tegmark and Dillon, Liu, \& Tegmark by estimating the foreground covariance from the data itself in a suitable constrained way. By optimizing the foreground downweighting beyond foreground covariance models we set a best limit of $\Delta(k) < 192$\,mK (95\%) at comoving scale $k = 0.18$\,$h$\,Mpc$^{-1}$ and $z = 6.8$, consistent with limits from other experiments. Motivated by wanting to extend the scientific return of a future 21\,cm detection, we perform a foreground and sensitivity analysis for 21\,cm--Ly$\alpha$ and 21\,cm--H$\alpha$ cross correlation experiments. Using 850\,nm observations of the MWA field with the Asteroid Terrestrial-impact Last Alert System (ATLAS), we demonstrate that the near infrared background fluctuations can be observed from the ground after careful image and fourier masking. We use the optimal quadratic estimator to set a first limit of $\Delta_{21,\text{Ly}\alpha}<13.3\,(\text{kJy/sr}\cdot \text{mK})^{1/2}$ (95\%) at $\ell=400$  on the broad band angular 21\,cm--Ly-$\alpha$ cross spectrum, and study what extensions are necessary to detect the EOR cross spectrum.
