% $Log: abstract.tex,v $
% Revision 1.1  93/05/14  14:56:25  starflt
% Initial revision
% 
% Revision 1.1  90/05/04  10:41:01  lwvanels
% Initial revision
% 
%
%% The text of your abstract and nothing else (other than comments) goes here.
%% It will be single-spaced and the rest of the text that is supposed to go on
%% the abstract page will be generated by the abstractpage environment.  This
%% file should be \input (not \include 'd) from cover.tex.

This thesis presents new instrumentation and analysis techniques for observations of redshifted 21\,cm and Lyman-$\alpha$, and H-$\alpha$  emission from the Epoch of Reionization (EOR), as well as early science results from their application to real data. We develop an all-sky, high dynamic range antenna measurement system and use it to characterize prototype antenna elements for the Murchison Widefield Array (MWA) and the Hydrogen Epoch of Reionization Array (HERA). We measure the model deviations at large zenith angles where the beam response crucially sets the amount of foreground power with high apparent frequency dependence in interferometer observations. We characterize the antenna-to-antenna variation over the MWA with lab measurements, and show that neglecting this effect results in severe foreground mis-subtraction.

We build on the optimal quadratic power spectrum analysis of Liu \& Tegmark and Dillon, Liu, \& Tegmark by estimating the covariance from the data itself in a suitably constrained way. We apply this method to 3\,hours of MWA data and set a limit on the 21\,cm EOR power spectrum of $\Delta(k) < 192$\,mK (95\%) at comoving scale $k = 0.18$\,$h$\,Mpc$^{-1}$ and $z = 6.8$, consistent with limits from other experiments. Then, to extend the scientific return of future 21\,cm detections, we perform a foreground and sensitivity analysis for 2D (i.e., broad band) 21\,cm--Ly$\alpha$ and 21\,cm--H$\alpha$ cross correlation experiments. Using data from the MWA, the Wide-field Infrared Survey Explorer (WISE), and the Asteroid Terrestrial-impact Last Alert System (ATLAS) we find percent-level flux correlations between radio and near-infrared foregrounds, consistent with geometric effects. We use the optimal quadratic estimator to set a first limit of $\Delta^2<181\,(\text{kJy/sr}\cdot \text{mK})$ (95\%) at $\ell=800$ on the broad band angular 21\,cm--Ly-$\alpha$ cross spectrum at $z\sim7$. We show that in the near term, higher resolution radio and near-infrared surveys such as LOFAR, Hubble, Dark Energy Survey can start to probe optimistic models of the 2D 21\,cm--Ly$\alpha$ EOR cross spectrum. 

