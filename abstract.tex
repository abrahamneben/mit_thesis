% $Log: abstract.tex,v $
% Revision 1.1  93/05/14  14:56:25  starflt
% Initial revision
% 
% Revision 1.1  90/05/04  10:41:01  lwvanels
% Initial revision
% 
%
%% The text of your abstract and nothing else (other than comments) goes here.
%% It will be single-spaced and the rest of the text that is supposed to go on
%% the abstract page will be generated by the abstractpage environment.  This
%% file should be \input (not \include 'd) from cover.tex.

This thesis presents new instrumentation and statistical techniques for observations of redshifted 21\,cm and Lyman-$\alpha$, and H-$\alpha$  emission from the Epoch of Reionization (EOR), as well as early science results from their application to real data. We develop a wide field, high dynamic range antenna measurement system and use it to characterize prototype antenna elements for the Murchison Widefield Array (MWA) and the Hydrogen Epoch of Reionization Array (HERA). We achieve measurements over 30\,dB of beam dynamic range over 65\% of the visible sky, far wider and deeper than drift scans through astronomical sources allow. We make the first measurements of model deviations at large zenith angles, where the beam response crucially sets the amount of foreground power with high fringe rate. These foregrounds are most at risk of contaminating cosmological measurements, and we quantify the implications of measured tens of percent model deviations for foreground avoidance and subtraction approaches. We extend our analysis of the MWA tile with precision laboratory measurements, finding overall gain and delay variation over the tile at the $\pm50$\,pm/$\pm0.5$\,dB level. Using these results, we perform the first study of antenna-to-antenna beam variation over the array. 

We then build on the optimal quadratic power spectrum analysis of Liu \& Tegmark and Dillon, Liu, \& Tegmark by estimating the foreground covariance from the data itself in a suitable constrained way. By optimizing the foreground downweighting beyond foreground covariance models we set a best limit of $\Delta^2(k) < (192 \text{ mK})^2$ (95\%) at comoving scale $k = 0.18$\,$h$\,Mpc$^{-1}$ and $z = 6.8$, consistent with limits from other experiments.

Beyond 21\,cm power spectrum studies, observations of the predicted large scale anticorrelation between 21\,cm and Ly-$\alpha$ emission could provide a cross-check on a purported 21\,cm detection and extend its science return by probing the sources driving ionized bubble growth. Working towards such correlation measurements,  we observe the MWA field at 850\,nm with the Asteroid Terrestrial-impact Last Alert System (ATLAS),  and demonstrate that with careful image and fourier masking the near infrared background can be observed from the ground. Using the best foreground subtraction and masking possible with these datasets, we use the optimal quadratic estimator to set the first limit on the broad band angular 21\,cm--Ly-$\alpha$ cross spectrum: $\Delta_{21,\text{Ly}\alpha}<13.3\,(\text{kJy/sr}\cdot \text{mK})^{1/2}$ (95\%) at $\ell=400$. We show that increasing 21\,cm resolution to $0.3'$ and improving foreground subtraction by a factor of $\sim10$ in power would permit a tenuous detection of the optimistic 21\,cm--Ly$\alpha$, and widening the infrared field of view to $40^\circ$ would yield significant detections at $\ell<1500$. 
