\chapter{Conclusion}

This thesis has presented experimental work towards observing redshifted 21\,cm, Ly-$\alpha$, and H-$\alpha$ emission from the EOR. I have discussed advances in instrumentation and analysis techniques, and applied them to real data, but much work remains to make direct observations of the cosmic dawn a reality. 
 
In Chapters \ref{chap:beampaper} and \ref{chap:herapaper}, I developed an antenna beam measurement system and made the first wide field, high dynamic range beam maps of MWA and HERA antenna elements. In Chapter \ref{chap:bferrors} I made the first detailed study of beamforming errors in an MWA tile, and modeled their effects on antenna-to-antenna variation over the array. For both experiments, I simulated the implications of the observed model deviations and found they do not leak power outside the EOR wedge, and thus do not interfere with foreground avoidance approaches. However, they do severely limit foreground subtraction fidelity within the wedge. Thus in situ, per-antenna beam measurements will be crucial to probe the relatively bright 21\,cm signal in this region of fourier space.

These results suggests that drone- and satellite-based beam measurements will be vital in the long run to recover the 21\,cm EOR signal in the foreground-contaminated wedge, but are not immediately necessary for first and second generation experiments seeking to measure the signal in the EOR window. Beam modeling errors will indeed limit sky-based calibration solutions, but \citet{ewallwice16b} argues that even if the beams were known perfectly, sky model uncertainties would still severely limit calibration. Indeed \citet{zheng14,ali15} have demonstrated with MITEoR and PAPER observations that redundant calibration is significantly more agnostic to sky and instrument modeling.

These results also bear on design of the upcoming Square Kilometer Array, whose current design consists of \citep{aavs,aavs2} roughly 500 phased array stations, each comprised of many log-periodic antennas scattered over a 30\,m diameter. Phased arrays are attractive because of their low costs relative to dishes, but our results have quantified the level of beamforming errors and concomitant beam distortions in a similar low cost phased array system. Before setting on a final tile design, it will be important to conduct similar laboratory measurements, generate monte carlo beams, and simulate foreground avoidance and subtraction to ensure that the EOR science is not compromised. The checkered history of the MWA has demonstrated that it is not enough to gamble that by collecting vast amounts of data, we will be able to solve precisely enough for all the instrument degrees of freedom. And even if we could, it is ideal to not have to learn this lesson by analyzing the science data.

On a related note, the temptation to position the stations pseudorandomly is great, as improving $uv$ coverage will improve image fidelity and lower the confusion limit. But \citet{beardsley16,ewallwice16b,barry16} warn of the limitations of sky-based approaches, and \citet{ali15} have demonstrated exquisite calibration and foreground isolation with redundant calibration. Indeed the MWA, whose antennas are positioned pseudorandomly, is in the process of adding redundant subarrays near its core antennas to mitigate uncalibratable cable reflections. 

Though the SKA looms high in the future, HERA is proceeding now, incorporating lessons from PAPER and the MWA. Its antennas are spaced on a grid like PAPER's, but they have large collecting area for sensitivity and horizon isolation like the MWA's. The main planned power spectrum pipeline builds on the per-baseline delay spectrum of PAPER \citep{deboer16,ewallwice16,neben16b,nithya16}, but foreground and instrumentation covariance must still be mitigated using an empirical covariance approach such as ours from Chapter \ref{chap:newlimits} \citep{parsons14,ali15}. 

Finally, as we discuss in Chapter \ref{chap:xcor}, observing the redshifted 21\,cm emission is only half the battle. Correlations with other observables are necessary to unravel the complex astrophysics of the EOR, and indeed convince ourselves that the results of unprecedented order thousand-hour averages are truly the cosmological signal. Along these lines, \citet{beardsley15} have proposed checking the 21\,cm brightness temperatures at the locations of the deepest JWST detections to study whether the most massive galaxies actually lie preferentially in the largest ionized regions. This per-source method, though, is limited to the brightest objects, and likely cannot probe the bulk of the ionizing population. 

In contrast, near-infrared intensity mapping probes the integrated light from entire ionizing population, and cross-correlation measurements against redshifted 21\,cm maps are predicted to show a distinct signature of the reionizing process: because the brightest galaxies should lie in the largest ionized bubbles, the correlation should be negative on scales larger than the ionized bubbles growing around galaxies, and positive on smaller scales where galactic overdensities stave off ionization. EOR galaxies are expected to be bright in emission lines such as Ly-$\alpha$ \citep{primevaltwins} and H-$\alpha$ \citep{brightemissionlines}, which redshift into the near- and mid-infrared.

We have emphasized that there are two paths to such cross correlation measurements: 2D and 3D observations. While 3D measurements are easier to relate to the astrophysics, they are severely sensitivity limited, not to mention the enormous cost of observing infrared cubes with even nearly comparable spectral resolution and field of view to 21\,cm cubes. In contrast, 2D (i.e., broad band) measurements are not limited by thermal or shot noise, meaning that the required infrared observations are substantially cheaper to conduct and the data are becoming available now. It is these 2D observations that we have focused on.

% but instead by radio and infrared foregrounds. This implies that a more pseudorandom tile positioning is actually preferable for the radio observations, though redundant subarrays would likely still facilitate calibration. 

Despite not being sensitivity limited, we have shown that 2D radio--infrared correlation measurements suffer large foreground contamination in two ways: (1) geometric effects produce percent-level correlations between radio and infrared foreground fluxes, even if their luminosities are independent; and (2) the uncorrelated component of radio and infrared fluxes results in a sample variance noise in correlation measurements. The former demands better foreground masking and subtraction, while the latter requires measurements over large fields of view with many independent samples to average down uncorrelated radio and infrared power.

Using data from ATLAS, we have shown that the airglow, mosaicing, and instrument systematics can be adequately mitigated to observe the fluctuations in the near-infrared background from the ground, permitting significant cost reductions. We then combined these observations with foreground-subtracted MWA observations to set the first limits on the 21\,cm--Ly$\alpha$ cross spectrum at $z\sim7$, and showed that higher resolution radio and infrared surveys such as LOFAR, SKA-LOW, Hubble, Dark Energy Survey can start to probe optimistic models of the 21\,cm--Ly$\alpha$ EOR cross spectrum.

Much more work on all fronts is needed to reach the EOR, but it represents the last gap in our observations of the universe between the smooth primordial universe following the big bang and the present day. And the prospects of untangling the astrophysics involved in forming stars and galaxies and ionizing the universe are sure to motivate us for decades to come.

