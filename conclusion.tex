\chapter{Conclusion}

This thesis has presented experimental work towards observing redshifted 21\,cm and ultraviolet emission from the EOR. Such observations are challenging, and though I have presented advances in instrumentation and analysis techniques, and applied them to real data, much work remains to make 21\,cm cosmology a reality. 
 
In Chapters \ref{chap:beampaper}, \ref{chap:bferrors}, and \ref{chap:herapaper}, I presented a prototype antenna beam measurement system and made the first wide field, high dynamic range beam maps of MWA and HERA antenna elements. I also made the first detailed study of beamforming errors in an MWA tile, and modeled their effects on antenna-to-antenna variation over the array. For both experiments, I simulated the implications of such model deviations and find find they do not leak power outside the EOR wedge, and thus do not interfere with foreground avoidance approaches. However, they do severely limit foreground subtraction fidelity in the wedge, and using in situ, per-antenna beam beam measurements will be necessary to achieve the exquisite subtraction needed to reveal foregrounds in the wedge. 

This suggests that drone- and satellite-based beam measurement techniques will be important for long-run efforts to recover the EOR signal in the foreground-contaminated wedge, but not immediately necessary to measure the signal in the foreground-free EOR window. Beam modeling errors will indeed limit fidelity of sky-based calibration solutions, but \citet{ewallwice16b} argues that even if the beams were known perfectly, sky model uncertainties severely limit sky-based calibration solutions. Indeed \citet{zheng14,ali15} have demonstrated with MITEoR and PAPER observations that redundant calibration is significantly more agnostic to sky and instrument modeling.

Our results also bear on design of the upcoming Square Kilometer Array, whose current design consists of \citep{aavs,aavs2} roughly 500 phased array tiles, each with many log-periodic antennas scatter over a 30\,m diameter. Phased arrays are attractive because of their low costs relative to dishes, but our results have quantified the level of beamforming errors and concomitant beam distortions in a low cost phased array system. Before setting on a final tile design, it will be important to conduct similar laboratory measurements, generate monte carlo beams, and simulate foreground avoidance and subtraction to ensure that the EOR science is not compromised. The checkered history of the MWA has demonstrated that it is not enough to gamble that by collecting vast amounts of data, we will be able to solve precisely enough for all the instrument degrees of freedom. And even if we could, it is ideal to not have to learn this lesson by analyzing the science data.

On a related note, the temptation to position the tiles pseudorandomly is great, as improving $uv$ coverage will improve image fidelity and lower the confusion limit. But \citet{beardsley16,ewallwice16b,barry16} warn of the limitations of sky-based approaches, and \citet{ali15} have demonstrate exquisite calibration and foreground isolation with redundant calibration. Indeed the MWA, whose antennas were originally positioned pseudorandomly, is in the process of adding redundant subarrays near its core antennas to study combined redundant/sky-model approaches. 

Though the SKA looms high in the future, HERA is going forward new, incorporating lessons from PAPER and the MWA. Its antennas are spaced on a grid like PAPER's, but they have large collecting area for sensitivity and horizon isolation like the MWA's. The main planned power spectrum pipeline builds on the per-baseline delay spectrum of PAPER \citep{deboer16,ewallwice16,neben16b,nithya16} , but foreground and instrumentation covariance must still be mitigated using an empirical covariance approach such as ours from Chapter \ref{chap:newlimits} \citep{parsons14,ali15}. 

Finally, as we discuss in Chapter \ref{chap:xcor}, observing the redshifted 21\,cm emission is only half the battle. Correlations with other observables are necessary to unravel the complex astrophysics of the EOR, and indeed convince ourselves that what results after hundred-hour averages is truly the cosmological signal, not one of manifold systematics. As we demonstrate, wide field near infrared measurements can be conducted from the ground if done with overlapping fields of view to mitigate airglow structure and mosaicinc patchiness, but the proposed SPHEREx mission will truly revolutionize this science with its all-sky, wide field maps. We synthesize our foreground measurements into the first analysis of residual foregrounds and the noise they generate in 21\,cm/IR cross correlation experiments, and demonstrate what class of experiments is needed to detect the signal.

Much more work on all fronts is needed to reach the EOR, but it represents the last gap in our observations of the universe between the minutes following the big bang and the present day. And the prospects of untangling the astrophysics involved in forming stars and galaxies and ionizing the universe are sure to motivate us for decades to come.

